%%%%%%%%%%%%%%%%%%%%%%%%%%%%%%%%%%%%%%%%%
% Beamer Presentation
% LaTeX Template
% Version 1.0 (10/11/12)
%
% This template has been downloaded from:
% http://www.LaTeXTemplates.com
%
% License:
% CC BY-NC-SA 3.0 (http://creativecommons.org/licenses/by-nc-sa/3.0/)
%
%%%%%%%%%%%%%%%%%%%%%%%%%%%%%%%%%%%%%%%%%

%----------------------------------------------------------------------------------------
%   PACKAGES AND THEMES
%----------------------------------------------------------------------------------------

\documentclass{beamer}

\usepackage[T2A]{fontenc}
\usepackage[utf8]{inputenc}
\usepackage[russian]{babel}
\usepackage{graphicx}
\usepackage{amssymb}
\usepackage{amsthm}
\usefonttheme{serif}
\usepackage{beamerthemePittsburgh}
\usepackage{color}

\mode<presentation> {

% The Beamer class comes with a number of default slide themes
% which change the colors and layouts of slides. Below this is a list
% of all the themes, uncomment each in turn to see what they look like.

%\usetheme{default}
%\usetheme{AnnArbor}
%\usetheme{Antibes}
%\usetheme{Bergen}
%\usetheme{Berkeley}
\usetheme{Berlin}
%\usetheme{Boadilla}
%\usetheme{CambridgeUS}
%\usetheme{Copenhagen}
%\usetheme{Darmstadt}
%\usetheme{Dresden}
%\usetheme{Frankfurt}
%\usetheme{Goettingen}
%\usetheme{Hannover}
%\usetheme{Ilmenau}
%\usetheme{JuanLesPins}
%\usetheme{Luebeck}
% \usetheme{Madrid}
%\usetheme{Malmoe}
%\usetheme{Marburg}
%\usetheme{Montpellier}
%\usetheme{PaloAlto}
%\usetheme{Pittsburgh}
%\usetheme{Rochester}
%\usetheme{Singapore}
%\usetheme{Szeged}
%\usetheme{Warsaw}

% As well as themes, the Beamer class has a number of color themes
% for any slide theme. Uncomment each of these in turn to see how it
% changes the colors of your current slide theme.

%\usecolortheme{albatross}
%\usecolortheme{beaver}
%\usecolortheme{beetle}
%\usecolortheme{crane}
%\usecolortheme{dolphin}
%\usecolortheme{dove}
%\usecolortheme{fly}
%\usecolortheme{lily}
%\usecolortheme{orchid}
%\usecolortheme{rose}
%\usecolortheme{seagull}
\usecolortheme{seahorse}
%\usecolortheme{whale}
%\usecolortheme{wolverine}

%\setbeamertemplate{footline} % To remove the footer line in all slides uncomment this line
%\setbeamertemplate{footline}[page number] % To replace the footer line in all slides with a simple slide count uncomment this line

%\setbeamertemplate{navigation symbols}{} % To remove the navigation symbols from the bottom of all slides uncomment this line
}

\usepackage{graphicx} % Allows including images
\usepackage{booktabs} % Allows the use of \toprule, \midrule and \bottomrule in tables
\graphicspath{ {./pics/} }
\theoremstyle{plain}
\newtheorem{defi}{Определение}
\newtheorem{lem}{Лемма}
\newtheorem{stm}{Утверждение}
\newtheorem{thm}{Теорема}
\newtheorem{rem}{Замечание}
\newtheorem{cor}{Corollary}

%----------------------------------------------------------------------------------------
%   TITLE PAGE
%----------------------------------------------------------------------------------------

\title[Поиск клик]{Использование псевдофизической симуляции для поиска клик в графе} % The short title appears at the bottom of every slide, the full title is only on the title page

\author{Нефедов Александр} % Your name
\institute[ЧелГУ] % Your institution as it will appear on the bottom of every slide, may be shorthand to save space
{
Челябинский Государственный Университет \\ % Your institution for the title page
\medskip
\textit{nefedov.alexander91@yandex.ru} % Your email address
}
\date{\today} % Date, can be changed to a custom date

\begin{document}

% \begin{frame}
% \titlepage % Print the title page as the first slide
% \end{frame}

\begin{frame}
  \begin{center}
    \tiny{МИНОБРНАУКИ РОССИИ} \\
    \footnotesize\textbf{<<Челябинский государственный университет>>}
    \vspace{2mm}

    \footnotesize{Математический факультет} \\
    \tiny{Кафедра вычислительной математики}
  % \end{center}
    \vspace{2mm}

  % \begin{center}
    {\footnotesize\textbf{Применение псевдофизической симуляции для поиска клик в графе}}
  \end{center}

  % \noindent
  % \begin{tabular*}{\textwidth}{@{\extracolsep{\fill}}lcl}
    % & \hfill & 
    \tiny{Выполнил студент} \\
    % \\
    % & & 
    \footnotesize{Нефедов Александр Юрьевич}\\
    % \\ % имя студента
    % & & 
    \tiny{академической группы МП-401}\\
    % \\
    % & & 
    % \\
  % \end{tabular*}

  \vspace{4mm}

  % \noindent
  % \begin{tabular*}{\textwidth}{@{\extracolsep{\fill}}lcl}
  %   & \hfill &  \\
  %   & & Научный руководитель \\
  %   & & Лепчинский Михаил Германович \\
  %   & & кандидат физико-математических наук \\
  %   & & доцент \\
  % \end{tabular*}
  \tiny{Научный руководитель} \\
    % \\
    % & & 
    \footnotesize{Лепчинский Михаил Германович}\\
    % \\ % имя студента
    % & & 
    \tiny{кандидат физико-математических наук}\\
    % \\
    % & & 
    \tiny{доцент} \\
    % \\
    % & & 
    % \footnotesize{Прикладная математика и информатика}\\
  \begin{center}
    \small\the\year
  \end{center}
\end{frame}

\begin{frame}{Клика}
  \begin{defi}
    \textbf{Полный граф} -- граф, в котором каждая вершина соединена ребром с другой
    вершиной.
  % \end{defi}

  % \begin{defi}
    \textbf{Подграф} -- граф, содержащий некоторое подмножество вершин данного графа и 
    некоторое подмножество соединяющих их рёбер.
  % \end{defi}

  % \begin{defi}
    \textbf{Клика} -- полный подграф данного графа.
  \end{defi}
\end{frame}

\begin{frame}{Применение клик}
  \begin{itemize}
    \item Социология
    \item Химия
    \item Биология
    \item Математика
  \end{itemize}
\end{frame}

\begin{frame}{Вариации задачи}
  \begin{itemize}
    \item
    найти наибольшую клику (клику с наибольшим количеством вершин);

    \item
    найти клику максимального веса во взвешенном графе;

    \item
    найти все максимальные клики (клики, которые нельзя расширить);

    \item
    поиск или проверка на существование в графе клики заданного размера;
  \end{itemize}
\end{frame}

\begin{frame}{Существующие алгоритмы}
  \begin{itemize}
    \item Tarjan \& Trojanowski (1977)
    \item Bron–Kerbosch (1973)
    \item Robson (2001)
  \end{itemize}
\end{frame}

\begin{frame}{Описание симуляции}
  % Нарисовать картинку с вершинами и равнодействующими силами
  Данные шаги выполняются в двойном цикле по всем вершинам
  \begin{enumerate}
    \item Для каждой пары вершин
    \begin{enumerate}
      \item Вычислить расстояние между вершинами
      \item Вычислить направление действия силы
      \item Рассчитать значение силы
    \end{enumerate}
    \item Рассчитать равнодействующую силу и применить её
    \item Сделать шаг симуляции
  \end{enumerate}

\end{frame}


\begin{frame}{Виды гравитации}
  $R$ -- Расстояние между вершинами
  \begin{itemize}
    \item Постоянная -- $const$
    \item Линейная -- $\frac{1}{R}$
    \item Классическая -- $\frac{1}{R^2}$
    \item Логарифмическая -- $\frac{1}{log(R + 1)}$
    \item Радикальная -- $\frac{1}{\sqrt R}$
    \item Ступенчатая
  \end{itemize}
\end{frame}

\begin{frame}{Использованные инструменты}
  \begin{itemize}
    \item С++
    \item Simple and Fast Multimedia Library
    \item Box2D
  \end{itemize}
\end{frame}

\begin{frame}{Рабочий экран}
  \begin{figure}[h]
    \centering
      % \caption{Граф из 150 вершин в конце симуляции}
    % \label{end_pic}
    \includegraphics[width=\textwidth]{Screenshot-Diploma-2}
  \end{figure}
\end{frame}

\begin{frame}{Выводы}
  Плюсы
  \begin{itemize}
    \item Относительная простота
    \item Скорость
    \item Корректно работает в абсолютном большинстве случаев
  \end{itemize}

  Минусы
  \begin{itemize}
    \item Возможен некорректный результат
    \item Много параметров
    \item Трудно оценить скорость сходимости
  \end{itemize}
\end{frame}

\begin{frame}{Возможное развитие}
  \begin{itemize}
    \item Переход в 3D
    \item Увеличение точности симуляции
    \item Публикация в рецензируемом журнале
  \end{itemize}
\end{frame}

\begin{frame}
  \begin{center}
  \LARGE{Спасибо за внимание}
  \end{center}
\end{frame}
\end{document} 
