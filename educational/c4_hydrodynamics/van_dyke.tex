\documentclass[12pt,a4paper,titlepage]{article}
\usepackage[utf8]{inputenc}
\usepackage[T2A]{fontenc}
\usepackage{amsmath,amssymb,amsopn,amsthm}
\usepackage{graphicx,import}
\usepackage{color}
\usepackage[english,russian]{babel}
\usepackage{array}
\usepackage{titlesec}
\usepackage[titletoc,toc,page]{appendix}
\usepackage{listings}
 \textwidth=17cm
 \textheight=26cm
 \topmargin=-1cm
 \oddsidemargin=0cm
 \headheight=0cm
 \newcommand{\sectionbreak}{\clearpage}
 \renewcommand{\appendixtocname}{Приложение}
 \renewcommand{\appendixpagename}{Приложение}
 
\lstset{
    breaklines=true,
    basicstyle=\ttfamily,
     postbreak=\raisebox{0ex}[0ex][0ex]{\ensuremath{\color{red}\hookrightarrow\space}}
}

\theoremstyle{definition}
\newtheorem{definition}{Definition}[section]
 
\theoremstyle{remark}
\newtheorem*{remark}{Remark}

\begin{document}
\begin{titlepage}
\begin{center}
  МИНОБРНАУКИ РОССИИ\\
  Федеральное государственное бюджетное образовательное учреждение\\ высшего образования\\

\bfseries \flqq Челябинский государственный университет\frqq \\
\bfseries (ФГБОУ ВО \flqq ЧелГУ\frqq) \\[0.7cm]

Математический факультет\\
Кафедра вычислительной механики и информационных технологий\\[3.4cm]
\large\bfseries РЕФЕРАТ\\[1cm]
\textit{\large\bfseries{Метод сращивания асимптотических разложений по Ван Дайку\\[2cm]}}

{\footnotesize
\begin{tabular}[t]{lllllllllllllllllll}
& & & & & & & & & & & & & & & & &   Выполнил студент       &   Нефедов А.Ю.   \\
& & & & & & & & & & & & & & & & &   академическая группа   &   МП-401        \\
& & & & & & & & & & & & & & & & &   курс                   &   IV            \\
& & & & & & & & & & & & & & & & &   форма обучения         &   очная         \\
& & & & & & & & & & & & & & & & &   направление подготовки &   Прикладная математика    \\
& & & & & & & & & & & & & & & & &                          &    и информатика              \\
& & & & & & & & & & & & & & & & &   подпись                &   \rule{3.3cm}{0.01cm}\\
& & & & & & & & & & & & & & & & &                          &                 \\
& & & & & & & & & & & & & & & & &   дата                   &   "\rule{0,7cm}{0.01cm}"\rule{1cm}{0.01cm} 2017 г.\\
& & & & & & & & & & & & & & & & &                          &                 \\
& & & & & & & & & & & & & & & & &                          &                 \\
& & & & & & & & & & & & & & & & &                          &                 \\
& & & & & & & & & & & & & & & & &                          &                 \\
& & & & & & & & & & & & & & & & &                          &                 \\
& & & & & & & & & & & & & & & & &   Научный руководитель   &   Дементьев О. Н.\\
& & & & & & & & & & & & & & & & &   Должность              &   профессор     \\
& & & & & & & & & & & & & & & & &                          &   кафедры ВМиИТ  \\
& & & & & & & & & & & & & & & & &   Ученая степень         &   доктор ф.-м.н.\\
& & & & & & & & & & & & & & & & &   Ученое звание          &                 \\
& & & & & & & & & & & & & & & & &                          &                 \\
& & & & & & & & & & & & & & & & &                          &                 \\
& & & & & & & & & & & & & & & & &   подпись                &  \rule{3.3cm}{0.01cm}\\
& & & & & & & & & & & & & & & & &                          &                 \\
& & & & & & & & & & & & & & & & &   дата                   &  "\rule{0,7cm}{0.01cm}"\rule{1cm}{0.01cm} 2017 г.     \\

\end{tabular}
\\[1.5cm]}

\small{Челябинск}\\
\small{2017}
\end{center}
\end{titlepage}
\tableofcontents
\section{Неоднородности прямого разложения}

  Прежде чем обсуждать детали метода сращиваемых асимптотических разложений, полезно составить представление о том, как возникают задачи особых возмущений.
  Что является источником возмущений?
  Можем ли мы предсказать, к задаче каких возмущений — регулярных или особых — приведет данная физическая проблема?

  Классический признак особого поведения знаком по теории Прандтля пограничного слоя: малый параметр является множителем при одной из старших производных в дифференциальных уравнениях. 
  Тогда в прямой схеме метода возмущений эта производная теряется в первом приближении, так что порядок дифференциальных уравнений снижается.
  В связи с этим должны быть отброшены одно или несколько граничных условий, и вблизи границ, на которых они были наложены, приближение нарушается. 
  Так бывает, за исключением того маловероятного случая, когда исходные граничные условия согласуются с упрощенными дифференциальными уравнениями.
  
  Часто оказывается полезным исследовать обыкновенные линейные дифференциальные уравнения как математическую моцель, показывающую существенные черты более сложных задач. 
  Простая модель, которая показывает потерю старшей производной в теории пограничного слоя, дана Фридрихсом \cite{2} в виде уравнения
  \begin{equation} \label{eq:moddd}
    \varepsilon \frac{d^2f}{dx^2} + \frac{df}{dx} = a, f(0) = 0, f(1) = 1
  \end{equation}
  с точным решением
  $$
    f(x, \varepsilon) = (1-a) \frac{1-e^{-x/\varepsilon}}{1-e^{-1/\varepsilon}} + ax
  $$

  Однако если положить $\varepsilon = 0$, то дифференциальное уравнение станет уравнением
  первого порядка,
  так что оба граничных условия не могут быть удовлетворены,
  если только не окажется случайно, что $a=1$.
  Точное решение показывает,
  что условие при $x = 0$ должно быть отброшено.
  Тогда приближенное решение для малых $\varepsilon$ будет
  $$
    f(x, \varepsilon) \sim (1-a) + ax.
  $$

  Как следует из рис. 5.1, это является хорошим приближением всюду,
  за исключением «пограничного слоя», где $x = O(\varepsilon)$.
  Если ввести увеличенную внутреннюю координату $X$, 
  соответствующую этой области, и положить
  $$
    f(x, \varepsilon) = F(X, \varepsilon), X = \frac{x}{\varepsilon}
  $$
  то исходная задача \ref{eq:moddd} преобразуется в следующую:
  $$
    \frac{d^2F}{dX^2} + \frac{dF}{dX} = a\varepsilon, F(0) = 0, F(\frac{1}{\varepsilon}) = 1.
  $$
  Если теперь положить $\varepsilon = 0$, то решением дифференциального уравнения,
  удовлетворяющим внутреннему граничному условию, будет функция $ 1 - e^{-X}$
  с произвольным множителем.
  Наложение внешнего граничного условия привело бы к обращению этого множителя в единицу.
  Однако точное решение показывает, что этот результат неверен.
  Внешнее граничное условие должно быть отброшено при отыскании внутреннего решения,
  так же как было отброшено внутреннее граничное условие для внешнего решения.
  Взамен этого внутреннее решение нужно срастить с внешним при помощи принципа сращивания.
  Таким образом, мы найдем равномерно пригодное первое приближение для малых $\varepsilon$

  $$
  {
    \displaystyle f(x)\sim{
      \begin{cases}
        1 - a + ax, \varepsilon \rightarrow 0, x > 0 \\
        (1 - a)(1 - e^{-X}), \varepsilon \rightarrow 0, X=\frac{x}{\varepsilon} 
      \end{cases}
      }
    }
  $$

  В задаче особых возмущений признак,
  характеризуемый потерей старшей производной,
  чаще всего отсутствует.
  Так, в теории тонкого профиля неоднородность обусловлена не дифференциальными уравнениями,
  а граничными условиями.
  Подобным же образом для вязкого течения при малых числах Рейнольдса
  всепроизводные высшего порядка в приближенных уравнениях Стокса сохраняются;
  неоднородность здесь связана скорее с бесконечной протяженностью жидкости.

\section{Роль составного и внутреннего разложений}
  Задача особых возмущений, как правило, включает две независимые длины.
  В результате прямое возмущенное решение в координатах,
  отнесенных к первичной характерной длине,
  оказывается неверным в области,
  где существенной является вторичная характерная длина.
  Вторичная характерная длина не всегда очевидна.
  Ясно, что это хорда для крыла большого удлинения,
  толщина для профиля с плоско затупленной передней кромкой,
  вязкая длина $\nu/U$ для течения при малых числах Рейнольдса.
  Однако при больших числах Рейнольдса это будет корень квадратный из 
  произведения вязкой и геометрической длин. 
  Для профиля с закругленной передней кромкой это будет не
  толщина, а радиус кромки, который пропорционален квадрату толщины,
  делённому на длину хорды.
  Обтекание заостренного профиля дозвуковым потоком является весьма тонкой задачей, 
  граничной между однородными и неоднородными задачами; 
  в ней размер области неоднородности экспоненциально мал и не может быть 
  непосредственно сопоставлен с каким-либо физическим размером. 
  Подобное замечание относится и к вихревому слою на слегка наклоненном конусе в 
  сверхзвуковом потоке.

  Непосредственное возмущенное решение дает асимптотическое разложение вида
  \begin{equation}
    \begin{split}
      f(x, y, z; \varepsilon) \sim \sum \delta_n(\varepsilon)f_n(x,y,z) \\
      \varepsilon \rightarrow 0 \textup{ при фиксированных } x, y, z.
    \end{split}
  \end{equation}
  Здесь $\delta_n(\varepsilon)$ образуют соответствующую асимптотическую последовательность, 
  $x, y, z$ — координаты, приведенные к безразмерному виду отнесением их к первичной 
  характерной длине.
  Это разложение применимо там, где функции $f_n$ регулярны.
  Эти функции приобретают особенность в какой-либо точке,
  если течение в ее окрестности определяется вторичной,
  а не первичной характерной длиной.
  Такая точка лежит в бесконечности, если вторичная длина является большей.
  Для того чтобы видоизмененное разложение было равномерно пригодно,
  оно должно зависеть также от координат,
  приведенных к безразмерному виду отнесением их к вторичной характерной длине.
  Поскольку отношение первичной и вторичной длин является функцией от $\varepsilon$, 
  это равносильно зависимости также и от $\varepsilon$.
  Таким образом, равномерно пригодное разложение должно иметь более сложный вид
  $$
    f(x, y, z; \varepsilon) \approx \sum \delta_n(\varepsilon)
    g_n(x,y,z;\varepsilon) \textup{ равномерно при } \varepsilon \rightarrow 0.
  $$

  Поскольку параметр возмущения $\varepsilon$ входит теперь как неявно в 
  функции $g_n$, так и явно в асимптотическую последовательность $\delta_n$,
  написанное выражение не является асимптотическим разложением в обычном смысле.
  Пусть оно называется \textit{составным разложением}.
  Такие разложения обсуждались в связи с задачами особых возмущений Эрдейи \cite{3},
  который назвал их «обобщенными асимптотическими разложениями».
  Имеются два возражения по поводу работы с составными разложениями.
  Во-первых, с ними трудно производить действия;
  очевидно, такая обычная операция, как приравнивание
  одинаковых степеней $\varepsilon$,
  должна быть пересмотрена и,
  как будет видно из дальнейшего,
  составные ряды не определяются однозначно.
  Во-вторых, они излишне сочетают в себе сложности как прямых разложений,
  так и области неоднородности.

  Проще изолировать трудности, связанные с неоднородностью,
  построением дополнительного \textit{внутреннего разложения},
  пригодного в окрестности этой неоднородности.
  Это достигается введением новых \textit{внутренних координат},
  которые имеют порядок единицы в области неоднородности.
  Тогда внутреннее разложение имеет вид
  \begin{equation}
    \begin{split}
      f(x,y,z;\varepsilon) = \sum \Delta_n(\varepsilon)F_n(X,Y,Z) \\
      \textup{ для } \varepsilon \rightarrow 0 \textup{ при фиксированных } X,Y,Z.
    \end{split}
  \end{equation}
  Внутренние переменные обозначены прописными буквами.
  Асимптотическая последовательность $\Delta_n(\varepsilon)$
  может отличаться от асимптотической последовательности $\delta_n(\varepsilon)$
  для внешнего разложения.
  Если область неоднородности представляет собой окрестность точки,
  лежащей на конечном расстоянии,
  то внутренние координаты $X, Y, Z$ обычно приведены к безразмерному виду
  при помощи вторичной характерной длины.
  Если неоднородность имеет место вдоль линии,
  как в теории пограничного слоя,
  то изменяется только координата,
  нормальная к этой линии.
  Для случая неоднородности на бесконечности координаты иногда должны быть
  растянуты введением различных функций от $\varepsilon$
  для разных направлений.
  Подобно внешнему разложению, внутреннее разложение является асимптотическим рядом
  в обычном понимании, так что к нему применимы обычные операции.

  Часто бывает полезно предварительно ввести и зависимое переменное,
  которое так же, как и независимое,
  имело бы порядок единицы во внутренней и внешней областях.
  Тогда главные члены в асимптотических последовательностях 
  $\Delta_n(\varepsilon)$ и $\delta_n(\varepsilon)$ будут равны единице.
  Степень растяжения в общем случае будет различной
  для зависимого и независимого переменных.
  Следуя Каплуну \cite{4}, а также Лагерстрому и Коулу \cite{5},
  можно формализовать этот процесс следующими определениями.

  \theoremstyle{definition}
  \begin{definition}
    \textit{Внешние переменные}: безразмерные независимые и зависимые переменные,
    основанные на первичных характерных величинах задачи.
  \end{definition}

  \theoremstyle{definition}
  \begin{definition}
    \textit{Внешний предел}:
    предел при стремлении параметра возмущения $\varepsilon$ к нулю
    при фиксированных значениях внешних переменных.
  \end{definition}

  \theoremstyle{definition}
  \begin{definition}
    \textit{Внешнее разложение}: асимптотическое разложение для
    $\varepsilon \rightarrow 0$ при фиксированных внешних переменных.
    В принципе получается из точного решения последовательным применением
    внешнего предельного перехода в соответствии с выбранной
    внешней асимптотической последовательностью.  
  \end{definition}

  \theoremstyle{definition}
  \begin{definition}
    \textit{Внутренние переменные}: безразмерные независимые и зависимые переменные,
    растянутые при помощи соответствующих функций $\varepsilon$ так,
    что они имеют порядок единицы в области неоднородности внешнего разложения.
  \end{definition}
  \theoremstyle{definition}
  \begin{definition}
    \textit{Внутренний предел}: предел для $\varepsilon \rightarrow 0$
    при фиксированных значениях внутренних переменных.
  \end{definition}

  \theoremstyle{definition}
  \begin{definition}
    \textit{Внутреннее разложение}: асимптотическое разложение для
    $\varepsilon \rightarrow 0$
    при фиксированных внутренних переменных.
    В принципе получается из точного решения последовательным применением
    внутреннего передельного перехода в соответствии с выбранной внутренней
    асимптотической последовательностью.
  \end{definition}

  \theoremstyle{definition}
  \begin{definition}
    \textit{Составное разложение}: какой-либо ряд,
    который сводится к внешнему разложению,
    когда он разлагается асимптотически для $\varepsilon \rightarrow 0$
    во внешних переменных,
    и к внутреннему разложению — вовнутренних переменных.
  \end{definition}

  Метод сращивания двух дополнительных
  асимптотических разложений сводит задачу особых возмущений
  к ее простейшим возможным элементам.
  Если тем не менее внутренняя задача первого приближения 
  окажется «невозможной», то можно предположить, что сама
  по себе задача является нерешаемой. 
  Например, очевидно, что распространение теории тонкого профиля
  на дозвуковой сжимаемый поток в случае 
  закругленной передней кромки приводит к внутренней задаче
  дозвукового обтекания параболического цилиндра, для
  которой общее решение неизвестно. 
  Вязкое обтекание
  заостренного профиля при больших числах Рейнольдса
  приводит к внутренней задаче вязкого обтекания
  полубесконечной плоской пластины, для которой существуют
  только частные решения.
  Одно из преимуществ 
  рассматриваемого способа состоит в том, что даже в этих
  «невозможных» ситуациях можно использовать численные
  результаты или даже экспериментальные данные для
  того, чтобы получить внутреннее решение.
  
\section{Описание программы}

\addcontentsline{toc}{section}{Список литературы}
\begin{thebibliography}{9}
    \bibitem{1} Ван Дайк М.
      Методы возмущений в механике жидкостей. М.: Мир, 1967.
    \bibitem{2} Фридрихс (Friedrichs K. O.)
      Theory of viscous fluids.
      Fluid Dynamics, Ch 4,
      Brown univ., 1942
    \bibitem{3} Эрдейи (Erd\'elyi A. )
      An expansion procedure for singular prturbations,
      Atti Accad.
      Sci. Torino, Cl. Sci. Fis. Mat. Nat., 95 (1961), 651-672.
    \bibitem{4} Каплун (Kaplun S.)
      The role of coordinate systems in boundary-layer theory,
      Z. angew. Math. Phys., 5 (1954), 111-135
    \bibitem{5} Лагерстром, Коул (Lagerstrom P. A., Cole J. D.)
      Examples illustrating expansion procedures for the Navier-Stokes equations,
      J. Rat. Mech. Anal., 4 (1955), 817-882
\end{thebibliography}

    \clearpage


\end{document}
