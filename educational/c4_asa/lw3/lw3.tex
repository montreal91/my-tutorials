\documentclass[12pt,a4paper,titlepage]{article}
\usepackage[utf8]{inputenc}
\usepackage[T2A]{fontenc}
\usepackage{amsmath,amssymb,amsopn,amsthm}
\usepackage{graphicx,import}
\usepackage{color}
\usepackage[english,russian]{babel}
\usepackage{array}
\usepackage{titlesec}
\usepackage[titletoc,toc,page]{appendix}
\usepackage{listings}
 \textwidth=17cm
 \textheight=26cm
 \topmargin=-1cm
 \oddsidemargin=0cm
 \headheight=0cm
 \newcommand{\sectionbreak}{\clearpage}
 \renewcommand{\appendixtocname}{Приложение}
 \renewcommand{\appendixpagename}{Приложение}
 
\lstset{
    breaklines=true,
    basicstyle=\ttfamily,
     postbreak=\raisebox{0ex}[0ex][0ex]{\ensuremath{\color{red}\hookrightarrow\space}}
}
\begin{document}
\begin{titlepage}
\begin{center}
  МИНОБРНАУКИ РОССИИ\\
  Федеральное государственное бюджетное образовательное учреждение\\ высшего образования\\

\bfseries \flqq Челябинский государственный университет\frqq \\
\bfseries (ФГБОУ ВО \flqq ЧелГУ\frqq) \\[0.7cm]

Математический факультет\\
Кафедра теории управления и оптимизации\\[3.4cm]
\large\bfseries ЛАБОРАТОРНАЯ РАБОТА\\[1cm]
\textit{\large\bfseries{Кластерный анализ данных\\[2cm]}}

{\footnotesize
\begin{tabular}[t]{lllllllllllllllllll}
& & & & & & & & & & & & & & & & &   Выполнил студент       &   Нефедов А.Ю.   \\
& & & & & & & & & & & & & & & & &   академическая группа   &   МП-401        \\
& & & & & & & & & & & & & & & & &   курс                   &   IV            \\
& & & & & & & & & & & & & & & & &   форма обучения         &   очная         \\
& & & & & & & & & & & & & & & & &   направление подготовки &   Прикладная математика    \\
& & & & & & & & & & & & & & & & &                          &    и информатика              \\
& & & & & & & & & & & & & & & & &   подпись                &   \rule{3.3cm}{0.01cm}\\
& & & & & & & & & & & & & & & & &                          &                 \\
& & & & & & & & & & & & & & & & &   дата                   &   "\rule{0,7cm}{0.01cm}"\rule{1cm}{0.01cm} 2017 г.\\
& & & & & & & & & & & & & & & & &                          &                 \\
& & & & & & & & & & & & & & & & &                          &                 \\
& & & & & & & & & & & & & & & & &                          &                 \\
& & & & & & & & & & & & & & & & &                          &                 \\
& & & & & & & & & & & & & & & & &                          &                 \\
& & & & & & & & & & & & & & & & &   Научный руководитель   &   Никитина С. А.\\
& & & & & & & & & & & & & & & & &   Должность              &   доцент     \\
& & & & & & & & & & & & & & & & &                          &   кафедры ТУиО  \\
& & & & & & & & & & & & & & & & &   Ученая степень         &   кандидат ф.-м.н.\\
& & & & & & & & & & & & & & & & &   Ученое звание          &                 \\
& & & & & & & & & & & & & & & & &                          &                 \\
& & & & & & & & & & & & & & & & &                          &                 \\
& & & & & & & & & & & & & & & & &   подпись                &  \rule{3.3cm}{0.01cm}\\
& & & & & & & & & & & & & & & & &                          &                 \\
& & & & & & & & & & & & & & & & &   дата                   &  "\rule{0,7cm}{0.01cm}"\rule{1cm}{0.01cm} 2017 г.     \\

\end{tabular}
\\[1.5cm]}

\small{Челябинск}\\
\small{2017}
\end{center}
\end{titlepage}
\tableofcontents
\section{Введение}
  \par
  Исходными данными для лабораторной работы является выборочные данные по РФ за 2012 год Российского мониторинга экономического положения и здоровья населения НИУ-ВШЭ (RLMS-HSE)», проводимого Национальным исследовательским университетом - Высшей школой экономики и ЗАО «Демоскоп» при участии Центра народонаселения Университета Северной Каролины в Чапел Хилле и Института социологии РАН.

  \par
  Целью лабораторной работы является проведение кластерного анализа исходных данных с помощью пакета R.

\section{Подкотовка данных к работе}
  Исходные данные представлены в Excel.
  Перед тем, как загружать их в рабочее пространство пакета R, необходимо:
  \begin{enumerate}
    \item Выделить данные своего варианта из общего массива данных.
    \item Подключить пакет $xlsx$.
  \end{enumerate}

  После чего можно загрузить данные в рабочее пространство пакета R с помощью команды
  \begin{lstlisting}
    data.table <- read.xlsx("my_data.xls", head=TRUE, sheetName="my_sheet")
  \end{lstlisting}

  Функция $read.xlsx$ сформирует таблицу на основе данных, которые находятся в файле $my\_data.xls$ и запишет в переменную $data.table$.
\clearpage

\section{Иерархический кластерный анализ}
  \par
  Суть иерархической кластеризации состоит в последовательном объединении меньших кластеров в большие.
  В начале работы алгоритма все объекты являются отдельными кластерами. На первом шаге наиболее похожие объекты объединяются в кластер.
  На последующих шагах объединение продолжается до тех пор, пока все объекты не будут составлять один кластер.

  \par
  Существует 2 вида иерархического анализа:
  \begin{enumerate}
    \item Классификация признаков
    \item Классификация объектов
  \end{enumerate}

  Чтобы проводить кластерный анализ в пакете R, необходимо подключить модуль cluster, это можно сделать с помощью команды:

  \begin{lstlisting}
    library(cluster)
  \end{lstlisting}

  Рассмотрим классификацию признаков.
  В исходных данных представлены следующие признаки: Зарплата, Образование, Стаж, Тип поселения, Численность населенного пункта, Возраст.
  Построим матрицу расстояний между признаками, используя евклидову метрику и стандартизацию, для этого используем функцию:
  \begin{lstlisting}
    data.tTable.dist = daisy(t(data.table), metric="euclidean", stand=TRUE)
  \end{lstlisting}
  Функция найдет матрицу расстояний и запишет её в $data.tTable.dist$, заметим, что к $data.table$ была применена функция $t$ – транспонирование, это сделано для того, чтобы анализировать именно признаки, а не объекты.
  Теперь можно применить иерархический метод.
  Они разделяются в зависимости от того, какое расстояние будет использовано между кластерами.
  В пакете R можно использовать следующие:

  \begin{itemize}
    \item $single$ – принцип «ближнего соседа»
    \item $complete$ – принцип «дальнего соседа»
    \item $average$ - принцип «средней связи» (невзвешенной)
    \item $mcquitty$ - принцип «взвешенной средней связи» (веса пропорциональны размерам кластеров - числу объектов в них)
    \item $centroid$ – определение расстояния между центрами кластеров (невзвешенными)
    \item $median$ - принцип определения расстояния между взвешенными центрами кластеров
  \end{itemize}
  Применяем иерархический метод, используем функцию:
  \begin{lstlisting}
    data.tTable.hist = hclust(data.tTable.dist, method="median")
  \end{lstlisting}
  Чтобы увидеть результат, который представляется в виде дендрограммы, нужно применить команду:

  \begin{lstlisting}
    plot(data.tTable.hist)
  \end{lstlisting}

  \begin{figure}
    \includegraphics{d1}
    \caption{Дендрограмма признаков}
    \label{fig:dend1}
  \end{figure}
  На Рис. \ref{fig:dend1} видно, что образуется 3 кластера: 1-Численность населенного пункта; 2-Зарплата, 3-Возраст, Стаж, Образование, Пол, Тип поселения. Расстояние между 3 и 2 от 100 до 150, между 2 и 1 от 30 до 50.

  Рассмотрим классификацию объектов.
  Возьмём следующие признаки у объектов: Возраст, Стаж, Образование, Пол, Тип поселения, так как данные признаки образуют единый кластер, а значит при исследовании объектов, все признаки будут в равных условиях.
  Построим матрицу расстояний между объектами, используя манхэттенскую метрику и стандартизацию, тем самым сбалансируем влияния признаков, для этого используем функцию:

  \begin{lstlisting}
    data.table.dist = daisy(data.table, metric="manhattan", stand=TRUE)
  \end{lstlisting}
  Применим иерархический метод по принципу определения расстояния между взвешенными центрами кластеров.

  \begin{lstlisting}
    data.table.hist = hclust(data.table.dist, method="median")
  \end{lstlisting}

  Построим полученную дендрограмму объектов.
  
  \begin{lstlisting}
    plot(data.table.hist)
  \end{lstlisting}

  \begin{figure}
    \includegraphics{d2}
    \caption{Дендрограмма объектов}
    \label{fig:dend2}
  \end{figure}
  Выдвинем гипотезу о том, что исходная выборка разбивается на 7 кластеров по объектам.
  \clearpage

\section{Последовательный кластерный анализ}
  Метод $k$-средних – это метод последовательного кластерного анализа, цель которого является разделение $m$ наблюдений на $N$ кластеров, при этом каждое наблюдение относится к тому кластеру, к центру которого оно ближе всего. 
  Теперь с помощью метода $k$-средних разобьём выборку на нужное количество кластеров. Используем функцию:

  \begin{lstlisting}
    data.table.clara = clara(data.table, N, stand=TRUE)
  \end{lstlisting}

  Где $N$ – количество кластеров.
  Качество разбиения на кластеры в методе к-средних определяется на основе графического анализа и по значениям следующих величин:

  \begin{enumerate}
    \item $max\_diss$ (максимальное отклонение от центра кластера)
    \item $av\_diss$ (среднее отклонение от центра кластера)
    \item $isolation$ (отношение максимального различия объекта в кластере и его медианой к минимальному различию между медианами кластеров: чем меньше это значение, тем лучше кластер обособлен)
  \end{enumerate}
  Применим метод для 5, 6, 7, 8 кластеров и сравним их.

  \subsection {Для пяти кластеров}
    \begin{lstlisting}
      data.table.clara = clara(data.table, 5, stand=TRUE)
      data.table.clara$clusinfo

            size  max_diss  av_diss isolation
       [1,]  144 16.599125 3.306575  1.679405
       [2,]  803  5.642071 2.250150  2.095720
       [3,]  282 18.509565 2.768429  5.995038
       [4,]  887  5.278125 2.364170  2.187541
       [5,]  884  4.956731 2.147399  2.054338

    \end{lstlisting}
  \subsection {Для шести кластеров}
    \begin{lstlisting}
      data.table.clara = clara(data.table, 6, stand=TRUE)
      data.table.clara$clusinfo

           size  max_diss  av_diss isolation
      [1,]  144 16.599125 3.306575  1.760105
      [2,]  744  5.329167 2.231746  1.675752
      [3,]  381 19.639935 2.417591  7.687630
      [4,]  371  4.653867 2.141633  1.788516
      [5,]  691  5.956548 2.284624  2.468716
      [6,]  669  3.627568 1.924944  1.503461
    \end{lstlisting}
  \subsection {Для семи кластеров}
    \begin{lstlisting}
      data.table.clara = clara(data.table, 7, stand=TRUE)
      data.table.clara$clusinfo

           size  max_diss  av_diss isolation
      [1,]  144 16.599125 3.306575  1.661582
      [2,]  669  4.954638 2.081936  1.658670
      [3,]  743  4.729901 1.948753  1.872120
      [4,]  581  5.367315 2.129058  1.831547
      [5,]  366  4.557417 2.060836  1.803850
      [6,]  166 17.965849 3.043212  5.614189
      [7,]  331  4.454758 1.997276  1.520145
    \end{lstlisting}
  \subsection {Для восьми кластеров}
    \begin{lstlisting}
      data.table.clara = clara(data.table, 8, stand=TRUE)
      data.table.clara$clusinfo

           size  max_diss  av_diss isolation
      [1,]  142 14.064690 3.134321  1.511480
      [2,]  706  5.208532 2.007432  1.861722
      [3,]  321  5.179820 1.931726  1.989256
      [4,]  543  4.930514 2.277342  1.893512
      [5,]  542  3.602249 1.696302  1.832725
      [6,]  327  4.287959 1.948572  1.389096
      [7,]  354  4.232253 1.733273  2.153253
      [8,]   65 16.216580 3.813900  3.215388

    \end{lstlisting}
  \subsection {Анализ}
    Возьмем среднеарифметические значения полученных величин и сравним их.
    \begin{center}
      \begin{tabular}{| l | l | l | l |}
        \hline
          Число кластеров & max\_diss & av\_diss & isolation \\ \hline
          5 & 3,8570864 & 1,721794 & 1,4690936\\ \hline
          6 & 3.783439 & 1,5932 & 1,404088\\ \hline
          7 & 3,552451 & 1,478506 & 1,452299\\ \hline
          8 & 3,529459 & 1,45355 & 1,46981\\
        \hline
      \end{tabular}
    \end{center}
    По таблице 1 выберем число кластеров, где isolation меньше всего, это строка с 6 кластерами.
    
    Выведем графическое представление кластеризации:
    \begin{lstlisting}
      plot(data.table, col = data.table.clara$clustering)
    \end{lstlisting}

    \begin{figure}
      \includegraphics{d3}
      \label{fig:dend3}
    \end{figure}

    По Рисунку 3 можно заметить, что никакой из признаков не стал ключевым при образовании кластеров.

\section{Заключение}
  Разбив с помощью метода $k$-средних на 6 кластеров и глядя на медианы кластеров, можно предположить, что малое число долгожителей, женщины образованнее мужчин.
\end{document}
