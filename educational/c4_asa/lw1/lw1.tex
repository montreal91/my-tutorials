\documentclass[12pt,a4paper,titlepage]{article}
\usepackage[utf8]{inputenc}
\usepackage[T2A]{fontenc}
\usepackage{amsmath,amssymb,amsopn,amsthm}
\usepackage{graphicx,import}
\usepackage{color}
\usepackage[english,russian]{babel}
\usepackage{array}
\usepackage{titlesec}
\usepackage[titletoc,toc,page]{appendix}
\usepackage{listings}
 \textwidth=17cm
 \textheight=26cm
 \topmargin=-1cm
 \oddsidemargin=0cm
 \headheight=0cm
 \newcommand{\sectionbreak}{\clearpage}
 \renewcommand{\appendixtocname}{Приложение}
 \renewcommand{\appendixpagename}{Приложение}
 
\lstset{
    breaklines=true,
    basicstyle=\ttfamily,
     postbreak=\raisebox{0ex}[0ex][0ex]{\ensuremath{\color{red}\hookrightarrow\space}}
}
\begin{document}
\begin{titlepage}
\begin{center}
  МИНОБРНАУКИ РОССИИ\\
  Федеральное государственное бюджетное образовательное учреждение\\ высшего образования\\

\bfseries \flqq Челябинский государственный университет\frqq \\
\bfseries (ФГБОУ ВО \flqq ЧелГУ\frqq) \\[0.7cm]

Математический факультет\\
Кафедра теории управления и оптимизации\\[3.4cm]
\large\bfseries ЛАБОРАТОРНАЯ РАБОТА\\[1cm]
\textit{\large\bfseries{Первичная обработка статистических данных\\[2cm]}}

{\footnotesize
\begin{tabular}[t]{lllllllllllllllllll}
& & & & & & & & & & & & & & & & &   Выполнил студент       &   Нефедов А.Ю.   \\
& & & & & & & & & & & & & & & & &   академическая группа   &   МП-401        \\
& & & & & & & & & & & & & & & & &   курс                   &   IV            \\
& & & & & & & & & & & & & & & & &   форма обучения         &   очная         \\
& & & & & & & & & & & & & & & & &   направление подготовки &   Прикладная математика    \\
& & & & & & & & & & & & & & & & &                          &    и информатика              \\
& & & & & & & & & & & & & & & & &   подпись                &   \rule{3.3cm}{0.01cm}\\
& & & & & & & & & & & & & & & & &                          &                 \\
& & & & & & & & & & & & & & & & &   дата                   &   "\rule{0,7cm}{0.01cm}"\rule{1cm}{0.01cm} 2017 г.\\
& & & & & & & & & & & & & & & & &                          &                 \\
& & & & & & & & & & & & & & & & &                          &                 \\
& & & & & & & & & & & & & & & & &                          &                 \\
& & & & & & & & & & & & & & & & &                          &                 \\
& & & & & & & & & & & & & & & & &                          &                 \\
& & & & & & & & & & & & & & & & &   Научный руководитель   &   Никитина С. А.\\
& & & & & & & & & & & & & & & & &   Должность              &   доцент     \\
& & & & & & & & & & & & & & & & &                          &   кафедры ТУиО  \\
& & & & & & & & & & & & & & & & &   Ученая степень         &   кандидат ф.-м.н.\\
& & & & & & & & & & & & & & & & &   Ученое звание          &                 \\
& & & & & & & & & & & & & & & & &                          &                 \\
& & & & & & & & & & & & & & & & &                          &                 \\
& & & & & & & & & & & & & & & & &   подпись                &  \rule{3.3cm}{0.01cm}\\
& & & & & & & & & & & & & & & & &                          &                 \\
& & & & & & & & & & & & & & & & &   дата                   &  "\rule{0,7cm}{0.01cm}"\rule{1cm}{0.01cm} 2017 г.     \\

\end{tabular}
\\[1.5cm]}

\small{Челябинск}\\
\small{2017}
\end{center}
\end{titlepage}
\tableofcontents
\section{Введение}
  \par
  Исходными данными для лабораторной работы является выборочные данные по РФ за 2012 год Российского мониторинга экономического положения и здоровья населения НИУ-ВШЭ (RLMS-HSE)», проводимого Национальным исследовательским университетом - Высшей школой экономики и ЗАО «Демоскоп» при участии Центра народонаселения Университета Северной Каролины в Чапел Хилле и Института социологии РАН.

  \par
  Целью данной лабораторной работы является проведение предварительного анализа выборки, определение ее основных характеристик, их интерпретации, подготовки данных для дальнейших исследований.
  Исследование проводится с помощью языка программирования для статистической обработки данных R.

\section{Подготовка данных к работе}
  Исходные данные представлены в Excel.
  Перед тем, как загружать их в рабочее пространство пакета R, необходимо:
  \begin{enumerate}
    \item Выделить данные своего варианта из общего массива данных.
    \item Подключить пакет $xlsx$.
    \item Подключить библиотеку $moments$.
  \end{enumerate}

  После чего можно загрузить данные в рабочее пространство пакета R с помощью команды
  \begin{lstlisting}
    data.table <- read.xlsx("my_data.xls", head=TRUE, sheetName="my_sheet")
  \end{lstlisting}

  Функция $read.xlsx$ сформирует таблицу на основе данных, которые находятся в файле $my\_data.xls$ и запишет в переменную $data.table$.
  Далее командами, аналогичными
  \begin{lstlisting}
    salaries = sort(data.table[[1]])
  \end{lstlisting}
  нужно разделить таблицу на сортированные столбцы. 
\clearpage

\section{Зарплата}
  \begin{figure}
    \includegraphics{hist_col_1}
    \caption{Гистограмма зарплат}
    \label{salary_hist}
  \end{figure}
  Чтобы получить гистограмму зарплат, необходимо выполнить команду.
  Результать работы команды можно видеть на рисунке \ref{salary_hist}
  \begin{lstlisting}
    hist(salaries)
  \end{lstlisting}


  Выборочное среднее
  \begin{lstlisting}
    mean(salaries)
    [1] 16850.9
  \end{lstlisting}

  Выборочная дисперсия
  \begin{lstlisting}
    var(salaries)
    [1] 182402990
  \end{lstlisting}

  Выборочное среднеквадратичное отклонение
  \begin{lstlisting}
    sd(salaries)
    [1] 13505.67
  \end{lstlisting}

  Медиана
  \begin{lstlisting}
    median(salaries)
    [1] 14000
  \end{lstlisting}

  Мода
  \begin{lstlisting}
    sort(unique(salaries))[which.max(salaries.t)]
    [1] 15000
  \end{lstlisting}

  Выборочный коэффициент ассиметрии
  \begin{lstlisting}
    skewness(salaries)
    [1] 3.773138
  \end{lstlisting}

  Эксцесс
  \begin{lstlisting}
    kurtosis(salaries)
    [1] 34.06756
  \end{lstlisting}

  Также для зарплат необходимо проверить гипотезы о законе распределения (нормальном и логнормальном).

  Прежде всего, необходимо найти квантиль распределения $\chi^2$.
  \begin{lstlisting}
    qchisq(0.01, length(salaries) - 1)
    [1] 2821.778
  \end{lstlisting}

  Далее выполняются тесты для проверки гипотез о нормальном и логнормальном законе распределения соответственно.
  \begin{lstlisting}
    > chisq.test(salaries, y=NULL, p = rep(1/length(salaries), length(salaries)))

    Chi-squared test for given probabilities

    data:  salaries
    X-squared = 32462755, df = 2999, p-value < 2.2e-16

    >
    > chisq.test(log(salaries), y=NULL, p = rep(1/length(salaries), length(salaries)))

    Chi-squared test for given probabilities

    data:  log(salaries)
    X-squared = 169.4347, df = 2999, p-value = 1
  \end{lstlisting}

  По результатам проведения теста, в первом случае значение $\chi^2 = 32462755$, что многократно превышает допустимое значение на несколько порядков, поэтому гипотезу о том, что зарплаты имеют нормальное распределение принимать нельзя.
  Во втором случае $\chi^2 = 169.4347$, что намного меньше полученного выше критического значения, поэтому гипотезу о том, что логарифмы зарплат распределены нормально (сами зарплаты имеют логнормальное распределение), необходимо принять.

  Чтобы найти доверительный интервал для выборочного среднего, необходимо выполнить следующую команду:

  \begin{lstlisting}
    > t.test(log(salaries), conf.level=0.99)

    One Sample t-test

    data:  log(salaries)
    t = 709.7113, df = 2999, p-value < 2.2e-16
    alternative hypothesis: true mean is not equal to 0
    99 percent confidence interval:
    9.451223 9.520122
    sample estimates:
    mean of x 
    9.485672
  \end{lstlisting}

\section{Образование}
  \begin{figure}
    \includegraphics{hist_education}
    \caption{Гистограмма образования}
    \label{education_hist}
  \end{figure}

  Гистограмма (рис \ref{education_hist})
  \begin{lstlisting}
    hist(education)
  \end{lstlisting}


  Выборочное среднее
  \begin{lstlisting}
    mean(education)
    [1] 16.84367
  \end{lstlisting}

  Выборочная дисперсия
  \begin{lstlisting}
    var(education)
    [1] 14.76948
  \end{lstlisting}

  Выборочное среднеквадратичное отклонение
  \begin{lstlisting}
    sd(education)
    [1] 3.843109
  \end{lstlisting}

  Медиана
  \begin{lstlisting}
    median(education)
    [1] 18
  \end{lstlisting}

  Мода
  \begin{lstlisting}
    sort(unique(education))[which.max(education.t)]
    [[1] 21
  \end{lstlisting}

  Выборочный коэффициент ассиметрии
  \begin{lstlisting}
    skewness(education)
    [1] -0.6632536
  \end{lstlisting}

  Эксцесс
  \begin{lstlisting}
    kurtosis(education)
    [1] 2.543895
  \end{lstlisting}

\section{Опыт работы}
  \begin{figure}
    \includegraphics{hist_experience}
    \caption{Гистограмма образования}
    \label{experience_hist}
  \end{figure}

  Гистограмма (рис \ref{experience_hist})
  \begin{lstlisting}
    hist(experience)
  \end{lstlisting}

  Выборочное среднее
  \begin{lstlisting}
    mean(experience)
    [1] 19.52033
  \end{lstlisting}

  Выборочная дисперсия
  \begin{lstlisting}
    var(experience)
    [1] 146.1917
  \end{lstlisting}

  Выборочное среднеквадратичное отклонение
  \begin{lstlisting}
    sd(experience)
    [1] 12.09097
  \end{lstlisting}

  Медиана
  \begin{lstlisting}
    median(experience)
    [1] 18
  \end{lstlisting}

  Мода
  \begin{lstlisting}
    sort(unique(experience))[which.max(experience.t)]
    [1] 18
  \end{lstlisting}

  Выборочный коэффициент ассиметрии
  \begin{lstlisting}
    skewness(experience)
    [1] 0.3174265
  \end{lstlisting}

  Эксцесс
  \begin{lstlisting}
    kurtosis(experience)
    [1] 2.312857
  \end{lstlisting}

\section{Численность населенного пункта}
  \begin{figure}
    \includegraphics{hist_population}
    \caption{Гистограмма численности населенного пункта}
    \label{population_hist}
  \end{figure}

  Гистограмма (рис \ref{population_hist})
  \begin{lstlisting}
    hist(population)
  \end{lstlisting}

  Выборочное среднее
  \begin{lstlisting}
    mean(population)
    [1] 903962.6
  \end{lstlisting}

  Выборочная дисперсия
  \begin{lstlisting}
    var(population)
    [1] 6.09288e+12
  \end{lstlisting}

  Выборочное среднеквадратичное отклонение
  \begin{lstlisting}
    sd(population)
    [1] 2468376
  \end{lstlisting}

  Медиана
  \begin{lstlisting}
    median(population)
    [1] 95854
  \end{lstlisting}

  Мода
  \begin{lstlisting}
    sort(unique(population))[which.max(population.t)]
    [1] 11503501
  \end{lstlisting}

  Выборочный коэффициент ассиметрии
  \begin{lstlisting}
    skewness(population)
    [1] 3.831788
  \end{lstlisting}

  Эксцесс
  \begin{lstlisting}
    kurtosis(population)
    [1] 16.43082
  \end{lstlisting}

\section{Возраст}
  \begin{figure}
    \includegraphics{hist_age}
    \caption{Гистограмма возрастов}
    \label{age_hist}
  \end{figure}

  Гистограмма (рис \ref{age_hist})
  \begin{lstlisting}
    hist(age)
  \end{lstlisting}

  Выборочное среднее
  \begin{lstlisting}
    mean(age)
    [1] 40.737
  \end{lstlisting}

  Выборочная дисперсия
  \begin{lstlisting}
    var(age)
    [1] 143.8931
  \end{lstlisting}

  Выборочное среднеквадратичное отклонение
  \begin{lstlisting}
    sd(age)
    [1] 11.99555
  \end{lstlisting}

  Медиана
  \begin{lstlisting}
    median(age)
    [1] 40
  \end{lstlisting}

  Мода
  \begin{lstlisting}
    sort(unique(age))[which.max(age.t)]
    [1] 34
  \end{lstlisting}

  Выборочный коэффициент ассиметрии
  \begin{lstlisting}
    skewness(age)
    [1] 0.3110832
  \end{lstlisting}

  Эксцесс
  \begin{lstlisting}
    kurtosis(age)
    [1] 2.295483
  \end{lstlisting}

\section{Заключение}
  Из проведённых исследований следует грустный вывод.
  Размеры зарплат распределены логнормально, а образование, опыт и возраст, нормально, следовательно, размер зарплат от этих показателей практически не зависит.
  Размер зарплат имеет большой эксцесс; это означает, что очень малое количество людей имеет очень большую зарплату и очень большое количество людей малую зарплату.
  Соотношение средней заработной платы и максимальной в выборке оставляет желать лучшего.
\end{document}
